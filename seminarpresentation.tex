\documentclass{beamer}

\usepackage{graphicx}
\usepackage{amsmath}
\usepackage{hyperref}
\usepackage{tikz} % For creating diagrams
\usepackage{booktabs} % For better tables

% Title page
\title{What are the Most Important Statistical Ideas of the Past 50 Years?}
\author{Faryal Fodderwala}
\date{\today}

\begin{document}

% Title Slide
\frame{\titlepage}

% Slide 1: Introduction
\begin{frame}{Introduction}
\begin{itemize}
    \item Overview of 8 significant statistical ideas from 1970 to 2021.
    \item Authors: Andrew Gelman and Aki Vehtari.
    \item Purpose: To provoke thought and discussion about modern statistical innovations and their impact on data science.
\end{itemize}
\end{frame}

% Slide 2: Authors' Background
\begin{frame}{Authors' Background}
\begin{itemize}
    \item \textbf{Andrew Gelman}:
        \begin{itemize}
            \item Professor of Statistics and Political Science, Columbia University.
            \item Renowned for Bayesian statistics and multilevel modeling.
        \end{itemize}
    \item \textbf{Aki Vehtari}:
        \begin{itemize}
            \item Professor of Computational Probabilistic Modeling, Aalto University.
            \item Focused on Bayesian computation and model assessment.
        \end{itemize}
\end{itemize}
\end{frame}

% Slide 3: Overview of the Paper
\begin{frame}{Overview of the Paper}
\begin{itemize}
    \item Timeframe: 1970 to 2021, focusing on the development of modern statistics.
    \item 8 statistical ideas selected based on their influence on statistical theory, computation, and applications.
    \item Emphasis on integrating computation with statistical modeling.
\end{itemize}
\end{frame}

% Slide 4: Counterfactual Causal Inference
\begin{frame}{Counterfactual Causal Inference}
\begin{itemize}
    \item Allows causal inference using observational data.
    \item Framework based on "potential outcomes" or "counterfactuals."
    \item Example: Studying the effect of NYC’s "Vision Zero" traffic policy using observational data.
\end{itemize}

\[
\text{Causal Effect: } Y(1) - Y(0)
\]
\begin{itemize}
    \item \( Y(1) \): Outcome if treated.
    \item \( Y(0) \): Outcome if untreated.
    \item Challenge: Only one outcome is observed.
\end{itemize}

\begin{block}{Real-World Connection}
NYC Open Data provides datasets on traffic accidents, enabling causal analysis of interventions like "Vision Zero."
\end{block}
\end{frame}

% Slide 5: Bootstrapping and Simulation-Based Inference
\begin{frame}{Bootstrapping and Simulation-Based Inference}
\begin{itemize}
    \item Introduced by Bradley Efron (1979).
    \item Resampling technique to estimate sampling distributions without assumptions about data distribution.
\end{itemize}

\begin{exampleblock}{Algorithm:}
\begin{enumerate}
    \item Resample the dataset with replacement.
    \item Compute the statistic of interest (e.g., mean).
    \item Repeat \( n \) times to estimate variability.
\end{enumerate}
\end{exampleblock}

\begin{block}{Example: NYC 311 Calls Data}
Use bootstrapping to estimate variability in the average response time for complaints across boroughs.
\end{block}
\end{frame}

% Slide 6: Overparameterized Models and Regularization
\begin{frame}{Overparameterized Models and Regularization}
\begin{itemize}
    \item High-dimensional models with more parameters than data points.
    \item Regularization prevents overfitting by adding penalties to the model:
    \[
    \text{LASSO: } \min \left( ||Y - X\beta||^2 + \lambda ||\beta||_1 \right)
    \]
\end{itemize}

\begin{block}{Example}
Neural networks for NYC Open Data crime prediction:
\begin{itemize}
    \item Regularization reduces noise and ensures generalizable predictions.
\end{itemize}
\end{block}
\end{frame}

% Slide 7: Bayesian Multilevel Models
\begin{frame}{Bayesian Multilevel Models}
\begin{itemize}
    \item Models hierarchical data with varying parameters at different levels.
    \item Example: Modeling housing prices across NYC boroughs.
\end{itemize}

\[
y_{ij} = \beta_0 + \beta_1 X_{ij} + u_j + \epsilon_{ij}
\]
\begin{itemize}
    \item \( u_j \): Random effect for borough \( j \).
    \item \( \epsilon_{ij} \): Error term for observation \( i \) in borough \( j \).
\end{itemize}

\begin{block}{Advantage}
Combines individual-level and group-level variability for improved estimates.
\end{block}
\end{frame}

% Slide 8: Generic Computation Algorithms
\begin{frame}{Generic Computation Algorithms}
\begin{itemize}
    \item Advances in algorithms like MCMC, EM, and variational inference.
    \item Enabled complex models and large-scale Bayesian analysis.
\end{itemize}

\begin{block}{Connection to NYC Open Data}
Use MCMC to model traffic flow patterns and predict congestion hotspots.
\end{block}
\end{frame}

% Slide 9: Adaptive Decision Analysis
\begin{frame}{Adaptive Decision Analysis}
\begin{itemize}
    \item Framework for making decisions during experiments.
    \item Application: Stopping clinical trials early for ethical reasons.
\end{itemize}

\begin{block}{Real-World Example}
In NYC public health studies, adaptive analysis helps evaluate the success of vaccination campaigns.
\end{block}
\end{frame}

% Slide 10: Robust Inference
\begin{frame}{Robust Inference}
\begin{itemize}
    \item Focuses on reliability under model misspecification.
    \item Example: Median-based estimators for income disparity in NYC.
\end{itemize}

\begin{block}{Key Insight}
Robust inference allows valid results even when data deviates from assumptions.
\end{block}
\end{frame}

% Slide 11: Exploratory Data Analysis (EDA)
\begin{frame}{Exploratory Data Analysis (EDA)}
\begin{itemize}
    \item Emphasizes visualization and insights over strict models.
    \item Examples: Trends in NYC Open Data on crime or health disparities.
\end{itemize}

\includegraphics[width=\textwidth]{example-plot.png} % Add your visualization here.
\end{frame}

% Slide 12: Connection to NYC Open Data
\begin{frame}{Connection to NYC Open Data}
\begin{itemize}
    \item Apply statistical methods to NYC datasets.
    \item Example: Visualize and analyze health disparities using robust inference and EDA.
\end{itemize}
\end{frame}

% Slide 13: Conclusions
\begin{frame}{Conclusions and Future Directions}
\begin{itemize}
    \item These statistical ideas are foundational to modern data analysis.
    \item Future: Integration of machine learning with causal inference.
    \item Importance of robust and interpretable models for real-world applications.
\end{itemize}
\end{frame}

% Slide 14: Questions
\begin{frame}{Questions?}
\begin{center}
    Thank you! Any questions?
\end{center}
\end{frame}

\end{document}
